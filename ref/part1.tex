\chapter{ глава } \label{chapt1}


\section{ секция} \label{sect1_1}

\textbf{жир} 

\subsection{саб секциии} \label{subsect1_1_1}

Первое приложение, которое я создал было простой игрой, названной Терраниа, где пользователь создавал существо и следил за его жизнью с помощью ежедневных рассылок почты. Приложение было написано на C++,что потребовало написание большого числа кода, ибо не было в те времена фреймворка .NET.

WEB  даёт разработчикам готовые решения для доставки контента через интернет. Всё что было необходимо сделать - обработать хранящиеся данные и отобразить статистику в виде готового отчего. Все страницы клиентского приложения генерировала программа на C++.

Поскольку интерфейс не требовал интерактива ( все действия пользователь совершал на стадии создания существа ),  системе приходилось лишь генерировать отчеты. Поскольку пользователи не совершали никакие действия сами ( система лишь генерировала отчеты ), множество пользователей могли работать одновременно. Малое количество пользователей не допускала ситуации, что два пользователя одновременно читают или пишут данные.

%12

Следующий раз я столкнулся в WEB-ориентированным приложения, когда мне потребовалось поправить вёрстку чата, основанном на  UBB. Все его данные хранились на диске Часть страниц чата генерировалось ``на лету'' из имеющихся данных, остальные были статичны и генерировались заранее, сохраняясь на диск. Тактика предварительно создание страниц до сих пор используется в системах, где мало пишут - много читают данные.







%============================================================================================================================

\clearpage