\chapter{ Введение } \label{chapt1}

До того как мы окунёмся в проектирование и программирование, мы должны определиться с соглашениями. Если вы уже разрабатывали web-приложения, то можете смело опустить эту главу.

\section{ Что такое web-приложения?} \label{sect1_1}

%19

Если вы читаете эту книжку, скорее всего вы хорошо понимаете что такое web-приложение. Необходимо понимать, что web-приложение это не просто сайт и не приложение, как обычная программа --- это что-то среднее, что включает себя элементы и того,и другого.

Если web-сайт является страницей с данными, то web-приложение содержит и данные и способ доставки этих данных. Если обычный сайт разделяет язык разметки и каскадные таблицы стилей, то архитектор web-приложения действительно разделяет данные: данные в web-приложениях не имеют ничего общего с разметкой. Когда необходимо отобразить пользователю данные, они извлекаются из баз данных и доставляются пользователю. Важно понимать, что они не обязательно доставляются с использованием HTML;  они могут быть доставлены с помощью PDF на электронную почту.

Web-приложения не имеют страниц, как web-сайты. Например, если web-приложение имеет 10 страниц, то добавление новых данных в базу данных может увеличить количество страниц, при этом нет нужны правит разметку или писать дополнительный код.Так например, пользовательский поиск может сгенерировать 100 страниц с результатами, и это вовсе не означает, что необходимо было создать каждую из ста страниц - достаточно лишь пары шаблонов и заданной логики для генерирования любого количества страниц на лету по заданным параметрам.

Средний пользователь не сможет отличить web-приложение от web-сайта. Лишь тот, кто редактирует данные будет отчетливо понимать, что перед ним web-приложение.

С использованием AJAX ( асинхроного JavaScript и XML ) модель взаимодействия изменилась. Если раньше после пользовательских изменений данные отправлялись на сервер и в ответ приходила новая страница, то сейчас можно в фоновом режиме отправить данные и сразу же получить ответ, без перезагрузки страницы.

Суть web-приложений со временем меняется. Уже пройден большой путь с начала использования интерактивных приложений в интернете. Современные технологии идут к тому, что бы данные доставлялись по сети и обрабатывались на клиенты, и web-приложения очень хорошо вписываются в эту модель, ведь в основе их лежат системы управления базами данных.


\section{Как вы строите web-приложения?} \label{sect1_2}

Для постройки web-приложения необходимо два главных компонента: техническая платформа и программное обеспечение. Для небольшого web-приложения будет достаточного выделенного хостинга с СУБД. Для небольших приложений нет необходимости думать о аппаратной составляющей. Но по мере расширения системы потребность в проектировании аппаратной части будет возрастать всё больше. В книге будет рассмотрено обе части web-приложений и их эффективное взаимодействие.

%21

Разработчики, кто сталкиваются с небольшими нагрузками могут подумать о целесообразности в проектировании аппаратной части, ведь можно использовать коробочные решения. Но не существует универсальных коробочных решений для многомиллионных приложений, где  каждое приложение имеет собственную логику. Именно поэтому каждый крупный проект создаёт свои собственные решения.








%============================================================================================================================

\clearpage