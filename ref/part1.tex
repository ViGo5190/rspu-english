\chapter{ Введение } \label{chapt1}

До того как мы окунёмся в проектирование и программирование, мы должны определиться с соглашениями. Если вы уже разрабатывали web-приложения, то можете смело опустить эту главу.

\section{ Что такое web-приложения?} \label{sect1_1}

%19

Если вы читаете эту книжку, скорее всего вы хорошо понимаете что такое web-приложение. Необходимо понимать, что web-приложение это не просто сайт и не приложение, как обычная программа --- это что-то среднее, что включает себя элементы и того,и другого.

Если web-сайт является страницей с данными, то web-приложение содержит и данные и способ доставки этих данных. Если обычный сайт разделяет язык разметки и каскадные таблицы стилей, то архитектор web-приложения действительно разделяет данные: данные в web-приложениях не имеют ничего общего с разметкой. Когда необходимо отобразить пользователю данные, они извлекаются из баз данных и доставляются пользователю. Важно понимать, что они не обязательно доставляются с использованием HTML;  они могут быть доставлены с помощью PDF на электронную почту.









%============================================================================================================================

\clearpage