\chapter*{Введение}							% Заголовок
\addcontentsline{toc}{chapter}{Введение}	% Добавляем его в оглавление


%:\section{ Введение} \label{sect1_1}

%\subsection{Preface} \label{subsect1_1_1}

Первое приложение, которое я создал было простой игрой, названной Терраниа, где пользователь создавал существо и следил за его жизнью с помощью ежедневных рассылок почты. Приложение было написано на C++,что потребовало написание большого числа кода, ибо не было в те времена фреймворка .NET.

WEB  даёт разработчикам готовые решения для доставки контента через интернет. Всё что было необходимо сделать - обработать хранящиеся данные и отобразить статистику в виде готового отчего. Все страницы клиентского приложения генерировала программа на C++.

Поскольку интерфейс не требовал интерактива ( все действия пользователь совершал на стадии создания существа ),  системе приходилось лишь генерировать отчеты. Поскольку пользователи не совершали никакие действия сами ( система лишь генерировала отчеты ), множество пользователей могли работать одновременно. Малое количество пользователей не допускала ситуации, что два пользователя одновременно читают или пишут данные.

%12

Следующий раз я столкнулся в WEB-ориентированным приложения, когда мне потребовалось поправить вёрстку чата, основанном на  UBB. Все его данные хранились на диске Часть страниц чата генерировалось ``на лету'' из имеющихся данных, остальные были статичны и генерировались заранее, сохраняясь на диск. Тактика предварительно создание страниц до сих пор используется в системах, где мало пишут - много читают данные. UBB был написан на скриптовом языке Perl, что давало ощутимый плюс в процессе разработки, так как не надо было тратить время на компиляцию исходных кодов.

После работы над UBB я основал сайт <<UBB Hackers>> для людей, которые тратили своё время на улучшение функциональности в чатах. 

UBB был неконкурентоспособен, потому его перенос на одну из целевых систем --- Windows --- затруднялся из-за не возможности переноса блокировки файлов. Например, если два пользователя одновременно отвечали на одну тему, то возникало состояние конкурирующих запросов и одновременной записи в файл, что повреждало и уничтожало данные. Кроме того, для высоконагруженных систем предварительная генерация страниц  с записью на диск блокировала файловую систему.


%13
MySQL 3 изменило огромное количество вещей в мире web приложений. До MySQL использование баз данных было сложной задачей. Широкораспространённый PHP 4, совместно с phpMyAdmin упростили жизни разработчикам: теперь не надо помнить все консольные команды для работы с базой, а так же больше нету необходимости в специальных программах для проектирования баз данных.

MySQL принес возможность одновременной записи и чтения данных, без разрушения данных. Разработчики получили возможность работать с большими массивами данных, быстро их обрабатывать.

Границы web приложений расширились для масштабирование, функциональности и совместимости. Бум публичных API позволил создавать сервис-ориентированные приложения.

Крупнейшие и популярнейшие web приложения на текущий момент, такие как  Flickr, MySpace и Wikipedia, обрабатывают миллионы запросов к базе данный в день, имеют огромные массивы данных. Пока Google остаётся моделью для подражания крупных web сервисов, маленькие создают новую модель приложений, названную WEB 2.0. Эта модель подразумевает интерактивную запись и чтения данных, открытый API.

%\subsection{О чем эта книга} \label{subsect1_1_2}

\textbf{О чем эта книга} \\

Эта книга описывает разработку web приложений: разработку программной и технической частей. Будут рассмотрены архитектуры приложений, практика разработки, технологии и работа инфраструктуры. В книге будет представлено много теории, но необходимо помнить, что если мы проектируем вещи которые не можем построить, мы не сможем узнать, когда спроектировали правильно.

%14

Эта книга не о программировании. В книге не будет кусочков кода, а те что встречаются - лишь примеры. Главное в книге --- технологии и подходы в разработке.

Б\'{о}льшая часть книги будет о том как спроектировать архитектуру приложения и построить для него инфраструктуру. Под инфраструктурой будем понимать  комбинацию технической платформы, программного обеспечения и практики в развертывании и разработки. Всё это позволить построить широко-масштабируемое приложение.

Наибольшая глава книги рассказывает о масштабировании приложений.  В это главе рассмотрена пара полезных техник с рекомендациями. В конце книги будут рассмотрены техники поддержания приложения в работе и мониторинга. Мониторинг гетерогенных систем требует больших навыков и понимая того, какие метрики необходимо отслеживать. В последней главе книги будут рассмотрены  технологии масштабирования данных с применением API в реалиях гетерогенных систем.

%\subsection{Что необходимо знать} \label{subsect1_1_3}
\textbf{Что необходимо знать}\\

Эта книга не предназначена для тех, кто проектирует своё первое масштабируемое приложение. Вы должны обладать хотя бы небольшим опытом в проектировании web приложений, а так же уметь управлять пользовательскими данными.

%15

Книга содержить определённое количество советов, для понимания которых необходимы навыки в программировании, так же понимания базовых принципов архитектуры фон Неймана. Умение пользоваться командной строкой Unix необходимо для лучшего понимания принципов работы приложения. 

В книге описываются четыре основных современных технологии, главной из которых является PHP --- основной язык всех примеров. Знание любых Си подобных языков поможет в понимании кода на PHP для тех, кто не знаком с этим языком программирования. Так же будет использоваться язык Perl, который очень удобен для написания  консольных скриптов.

%16

Основной системой управления баз данных для всех примеров будет MySQL. Благодаря простой установке и настройке, а так же своей бесплатности, эта СУБД используется в огромном количестве web приложений. Для поддержания Unix окружения будет использоваться web-сервер Apache.

Умение использовать данное ПО не является самым необходимым: необходим опыт работы с большим стеком технологий, такими как HTTP, TCP/IP, MIME и другими. Для лучшего понимания книги необходимо ознакомиться со всеми этими стандартами и протоколами.







%============================================================================================================================

\clearpage