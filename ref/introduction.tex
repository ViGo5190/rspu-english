\chapter*{Введение}							% Заголовок
\addcontentsline{toc}{chapter}{Введение}	% Добавляем его в оглавление


%:\section{ Введение} \label{sect1_1}

%\subsection{Preface} \label{subsect1_1_1}

Первое приложение созданное автором было простой игрой, названной Терраниа, где пользователь создавал существо и следил за его жизнью с помощью ежедневных рассылок почты. Приложение было написано на C++,что потребовало написание большого числа кода, ибо не было в те времена фреймворка .NET.

WEB  даёт разработчикам готовые решения для доставки контента через интернет. Всё что было необходимо сделать - обработать хранящиеся данные и отобразить статистику в виде готового отчего. Все страницы клиентского приложения генерировала программа на C++.

Поскольку интерфейс не требовал интерактива ( все действия пользователь совершал на стадии создания существа ),  системе приходилось лишь генерировать отчеты. Поскольку пользователи не совершали никакие действия сами ( система лишь генерировала отчеты ), множество пользователей могли работать одновременно. Малое количество пользователей не допускала ситуации, что два пользователя одновременно читают или пишут данные.

%12

Следующий раз автор столкнулся в WEB-ориентированным приложения, когда ему потребовалось поправить вёрстку чата, основанном на  UBB. Все его данные хранились на диске Часть страниц чата генерировалось ``на лету'' из имеющихся данных, остальные были статичны и генерировались заранее, сохраняясь на диск. Тактика предварительно создание страниц до сих пор используется в системах, где мало пишут - много читают данные. UBB был написан на скриптовом языке Perl, что давало ощутимый плюс в процессе разработки, так как не надо было тратить время на компиляцию исходных кодов.

После работы над UBB автор основал сайт <<UBB Hackers>> для людей, которые тратили своё время на улучшение функциональности в чатах. 

UBB был неконкурентоспособен, потому его перенос на одну из целевых систем --- Windows --- затруднялся из-за не возможности переноса блокировки файлов. Например, если два пользователя одновременно отвечали на одну тему, то возникало состояние конкурирующих запросов и одновременной записи в файл, что повреждало и уничтожало данные. Кроме того, для высоконагруженных систем предварительная генерация страниц  с записью на диск блокировала файловую систему.


%13
MySQL 3 изменило огромное количество вещей в мире web приложений. До MySQL использование баз данных было сложной задачей. Широкораспространённый PHP 4, совместно с phpMyAdmin упростили жизни разработчикам: теперь не надо помнить все консольные команды для работы с базой, а так же больше нету необходимости в специальных программах для проектирования баз данных.

MySQL принес возможность одновременной записи и чтения данных, без разрушения данных. Разработчики получили возможность работать с большими массивами данных, быстро их обрабатывать.

Границы web приложений расширились для масштабирование, функциональности и совместимости. Бум публичных API позволил создавать сервис-ориентированные приложения.

Крупнейшие и популярнейшие web приложения на текущий момент, такие как  Flickr, MySpace и Wikipedia, обрабатывают миллионы запросов к базе данный в день, имеют огромные массивы данных. Пока Google остаётся моделью для подражания крупных web сервисов, маленькие создают новую модель приложений, названную WEB 2.0. Эта модель подразумевает интерактивную запись и чтения данных, открытый API.

%\subsection{О чем эта книга} \label{subsect1_1_2}

\textbf{О чем эта книга} \\

Эта книга описывает разработку web приложений: разработку программной и технической частей. Будут рассмотрены архитектуры приложений, практика разработки, технологии и работа инфраструктуры. В книге будет представлено много теории, но необходимо помнить, что если мы проектируем вещи которые не можем построить, мы не сможем узнать, когда спроектировали правильно.

%14

Эта книга не о программировании. В книге не будет кусочков кода, а те что встречаются - лишь примеры. Главное в книге --- технологии и подходы в разработке.

Б\'{о}льшая часть книги будет о том как спроектировать архитектуру приложения и построить для него инфраструктуру. Под инфраструктурой будем понимать  комбинацию технической платформы, программного обеспечения и практики в развертывании и разработки. Всё это позволить построить широко-масштабируемое приложение.

Наибольшая глава книги рассказывает о масштабировании приложений.  В это главе рассмотрена пара полезных техник с рекомендациями. В конце книги будут рассмотрены техники поддержания приложения в работе и мониторинга. Мониторинг гетерогенных систем требует больших навыков и понимая того, какие метрики необходимо отслеживать. В последней главе книги будут рассмотрены  технологии масштабирования данных с применением API в реалиях гетерогенных систем.

%\subsection{Что необходимо знать} \label{subsect1_1_3}
\textbf{Что необходимо знать}\\

Эта книга не предназначена для тех, кто проектирует своё первое масштабируемое приложение. Вы должны обладать хотя бы небольшим опытом в проектировании web приложений, а так же уметь управлять пользовательскими данными.

%15

Книга содержить определённое количество советов, для понимания которых необходимы навыки в программировании, так же понимания базовых принципов архитектуры фон Неймана. Умение пользоваться командной строкой Unix необходимо для лучшего понимания принципов работы приложения. 

В книге описываются четыре основных современных технологии, главной из которых является PHP --- основной язык всех примеров. Знание любых Си подобных языков поможет в понимании кода на PHP для тех, кто не знаком с этим языком программирования. Так же будет использоваться язык Perl, который очень удобен для написания  консольных скриптов.

%16

Основной системой управления баз данных для всех примеров будет MySQL. Благодаря простой установке и настройке, а так же своей бесплатности, эта СУБД используется в огромном количестве web приложений. Для поддержания Unix окружения будет использоваться web-сервер Apache.

Умение использовать данное ПО не является самым необходимым: необходим опыт работы с большим стеком технологий, такими как HTTP, TCP/IP, MIME и другими. Для лучшего понимания книги необходимо ознакомиться со всеми этими стандартами и протоколами.


%
%\section{ Что такое web-приложения?} \label{sect1_1}

\textbf{Что такое web-приложения?}\\

%19

Если вы читаете эту книжку, скорее всего вы хорошо понимаете что такое web-приложение. Необходимо понимать, что web-приложение это не просто сайт и не приложение, как обычная программа --- это что-то среднее, что включает себя элементы и того,и другого.

Если web-сайт является страницей с данными, то web-приложение содержит и данные и способ доставки этих данных. Если обычный сайт разделяет язык разметки и каскадные таблицы стилей, то архитектор web-приложения действительно разделяет данные: данные в web-приложениях не имеют ничего общего с разметкой. Когда необходимо отобразить пользователю данные, они извлекаются из баз данных и доставляются пользователю. Важно понимать, что они не обязательно доставляются с использованием HTML;  они могут быть доставлены с помощью PDF на электронную почту.

Web-приложения не имеют страниц, как web-сайты. Например, если web-приложение имеет 10 страниц, то добавление новых данных в базу данных может увеличить количество страниц, при этом нет нужны правит разметку или писать дополнительный код.Так например, пользовательский поиск может сгенерировать 100 страниц с результатами, и это вовсе не означает, что необходимо было создать каждую из ста страниц - достаточно лишь пары шаблонов и заданной логики для генерирования любого количества страниц на лету по заданным параметрам.

Средний пользователь не сможет отличить web-приложение от web-сайта. Лишь тот, кто редактирует данные будет отчетливо понимать, что перед ним web-приложение.

С использованием AJAX ( асинхроного JavaScript и XML ) модель взаимодействия изменилась. Если раньше после пользовательских изменений данные отправлялись на сервер и в ответ приходила новая страница, то сейчас можно в фоновом режиме отправить данные и сразу же получить ответ, без перезагрузки страницы.

Суть web-приложений со временем меняется. Уже пройден большой путь с начала использования интерактивных приложений в интернете. Современные технологии идут к тому, что бы данные доставлялись по сети и обрабатывались на клиенты, и web-приложения очень хорошо вписываются в эту модель, ведь в основе их лежат системы управления базами данных.


%\section{Как вы строите web-приложения?} \label{sect1_2}
\textbf{Как вы строите web-приложения?}\\

Для постройки web-приложения необходимо два главных компонента: техническая платформа и программное обеспечение. Для небольшого web-приложения будет достаточного выделенного хостинга с СУБД. Для небольших приложений нет необходимости думать о аппаратной составляющей. Но по мере расширения системы потребность в проектировании аппаратной части будет возрастать всё больше. В книге будет рассмотрено обе части web-приложений и их эффективное взаимодействие.

%21

Разработчики, кто сталкиваются с небольшими нагрузками могут подумать о целесообразности в проектировании аппаратной части, ведь можно использовать коробочные решения. Но не существует универсальных коробочных решений для многомиллионных приложений, где  каждое приложение имеет собственную логику. Именно поэтому каждый крупный проект создаёт свои собственные решения.

Как уже было сказано в основе web-приложений лежит СУБД. Кроме выбора самой СУБД необходимо решить вопросы хранения данных, доступа и редактирования, а так же представления этих данных. Во второй главе книге будет рассмотрены различные компоненты и способы их взаимодействия.

Основная цель книги научить проектировать и создавать расширяемые приложения. К концу книге Вы будете хорошо понимать как проектировать приложение и архитектуру, как расширять свою систему, как расширять и применять эти наработки.



%\section{Что такое архитектура?} \label{sect1_3}
\textbf{Что такое архитектура?}\\

Мы так много говорим об архитектуре приложения, что возникает вопрос: что это такое? Когда архитектор проектирует дом, он имеет хорошо поставленные задачи: собрать требования, изучить возможности и предоставить план. Когда план превращается в строение, мы ожидаем  несколько вещей: здание должно  стоять, выдерживать дождь и ветер и быть достаточно светлым. Разрушая иллюзии, необходимо понимать, что проектирование web-приложение - совсем другое.

%22

Если бы дома строились как приложения, то архитектор принимал бы непосредственное участие в строительном процессе c закладки фундамента и установки арматуры.  Он бы начал проектировать с пары комнат, в которые бы вселились постояльцы ещё до окончания стройки. По мере стройки всё новые доступные площади будут заниматься новыми жильцами. Но эти новые жильцы будут требовать доработок: новых спален, изменения планировки и прочего. И все эти правки придется вносить в текущий проект. При этом на время постройки и доработки, жильцы не будут покидать свой дом, при этом будут постоянно ныть про шум и и грязь от стройки, а так же о том что всё очень долго. При этом жильцы будут сразу же заселять все новые постройки и расширения. К моменту окончания стройки, появятся новые жильцы, которых надо будет где-то заселить и сделать счастливыми.

Хороший архитектор должен предусмотреть все возможные варианты. Но постройка сразу огромного дома может затянуться, а слишком скромного может не удовлетворить желания всех. Посему необходимо на начальном этапе определить масштабы и возможности.

Никто не говорить, что с первого раза всё должно быть правильно. Всегда придется пересматривать принятые решения и переделывать текущие реализации. Задача архитектора приложения --- минимизировать время на переработку текущих решений и аккуратно внести изменения.

%\section{С чего начать?} \label{sect1_3}
\textbf{С чего начать?}\\

Для того, чтобы начать строить приложения необходимо иметь идея, что является не типичной деятельностью для инженера. Если у вас есть небольшой проект, который растет или даже если у вас есть огромный проект - книга для вас.

Для построения малых и средних проектов требуется лишь один инженер, но для огромных проектов, где тысячи строк кода, необходимо большое количество сотрудников, которыми необходимо управлять, а так же предоставить им все средства для разработки и тестирования проекта. 

%23

Необходимо документировать идеи, планы и способы реализации внутри проекта. Хорошая документация в большом проекте помогает соблюдать принятые стандарты. Главное знать меру в документировании и не превращать это в бюрократию.

Разработка web-приложений более итеративна, чем классические методы разработки.  При проектировании хочется избежать шагов, которые заведут разработку в тупик, посему короткие этапы разработки помогают быстрее оценить ошибки и вернуться на шаг назад. Написание документации должно идти совместно с этапами разработки, не опережая её и не запаздывая.






%============================================================================================================================

\clearpage