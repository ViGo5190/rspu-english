\chapter{ Интернализация, локализация и Unicode } \label{chapt4}

%87

Мир говорит на разных языках. Поэтому создавая приложения на одном языке сразу же теряется очень большая часть возможных пользователей.

Задача интернализации сложна и требует применения определенных знаний и методов.

\section{ Интернализация и локализация} \label{sect4_1}

 %88
 
 Интернализация приложения --- создание приложения, которое может быть использовано людьми говорящих на разных языках. Локализация приложения --- изменения приложения согласно тому где находиться пользователь.
 
 \section{ Интернализация web-приложений} \label{sect4_2}
 
 В девяностых годах прошлого тысячелетия была огромная проблема с интернализацией приложений. Каждый язык использовал свою кодировку. Поскольку для разных языков кодировка отличалась кодом букв в таблице ASCII ( между 100 и 250 ), то вывести на одной странице набор данных в разных кодировках было невозможно ( один и тот же код в разных кодировках означала разный символ).
 
 %89
 
 В 1991 году был создан Unicode --- кодировка которая содержит символы всех остальных кодировок. Таким образом один код всегда означает один символ. Существуют две основные кодировке на базе Unicode: UTF-8 и UTF-16 ( расширенная ).
 
\section{Локализация в web-приложения} \label{sect4_3}

Локализация подразумевает отображение разного контента для различных людей, согласно их местоположению.

%90

\subsection{Замена строк} \label{sect4_3_1}

Первый метод интернализации заключается в замене текста на одном языке на текст в другом языке. Для этого используется GNU приложение gettext.

\subsection{Множественные шаблоны} \label{sect4_3_2}

При правильном разделении логики и представления, можно создавать множественные шаблоны ( для каждой интернализации свой набор шаблонов). Таким образом сохраняя один и тот же функционал, можно использовать разное представления для разных людей.

%91

\subsection{Множественный интерфейс} \label{sect4_3_3}

Использование собственного интерфейса для каждого пользователя ( интернализации ) подразумевает собственную бизнес логику, логику представления и шаблоны. Таким образом  придется создавать часть приложения для каждого пользователя. Так же необходимо решить, будут ли данные пересекаться между интернализациями или каждая будет иметь свой набор уникальных данных.

\section{В двух словах о Unicode} \label{sect4_4}

Unicode содержит в себе всевозможные символы, их написание и начертание. Коды базовых написаний латинских букв в Unicode совпадают с кодами написания тех же букв в ASCII. 

%93

\section{Кодирование Unicode} \label{sect4_5}

В Unicode символ кодируется словом либо фиксированной длины ( одинаковое количество байт ), либо плавающей. UTF-16 кодируется 4 байтам, UCS2 --- 2 байтами,  а UTF-8 ---от 1 до 4 байт.

%94

Необходимо понимать, что разные кодировки --- всего лишь разные правила отображения кода символа в изображение символа. А Unicode предоставляет единое отображение кода символа в изображение символа.

\section{Код символа, его значек, изображения и графема} \label{sect4_6}

Один и тот же символ может быть отображен разным кодом. Но разный код подразумевает разные вещи. Так можно отобразим просто букву, букву с ударением или графему. В некоторых языках принято писать <<склеивая>> буквы, такой символ может быть отображен с помощью двух кодов ( буквы будут рядом) или одного специального кода ( букву будут <<склеины>> ).

%95

Важно понимать, что получив строку в UTF-8 нельзя узнать её длину по количеству байт, ведь длина кода не фиксирована.

%96

\section{Byte Order Mark (BOM) } \label{sect4_7}

BOM --- последовательность байтов в начале каждого Unicode  файла описывающая способ декодирования. Таким образом, прочитав эти данные в начале файле система поймет в какой кодировке дальше будут файлы.

%97

К некоторых документах наличие BOM может сломать логику приложения, поэтому необходимо удалять BOM в HTML, XML и PHP файлах.

\section{ Кодирование UTF-8 } \label{sect4_8}

UTF-8 наиболее распространенная кодировка в мире web-приложений. Благодаря не фиксированной длине кода и ориентированности на латинские символы, она позволяет экономить места ( по сравнению с UTF-16 ) в английских текстах.

%98

Одно из главных преимуществ UTF-8 --- возможность использования кодировки в системах с прямым и обратным порядком байт без конвертирования данных.

\section{ UTF-8 в web-приложениях } \label{sect4_9}

\subsection{Управление выводом } \label{sect4_9_1}

Для использование UTF-8 в приложении всего его данные, а так же исходный код должны быть конвертированы в UTF-8. Так же необходимо использовать средства разработки, которые поддерживают работу с UTF-8. В заголовке ответа сервера необходимо так же указать, что ответ будет передан в кодировке UTF-8.

 %99
 
Так же, для большей надёжности, можно указать кодировку в специальных тегах разметки. Это позволит избежать проблем с отображением у тех, кто решит сохранить страницу приложения и открыть её после.

\subsection{Управление входными данными } \label{sect4_9_2}

%100

Наивно полагать, что все входные данные будут в UTF-8. Многие браузеры могут передать данные в той кодировке, которая установлена у пользователя по умолчанию. Поэтому необходимо проверять входные данные на кодировку, и, если данные пришли кодировке отличной от UTF-8, конвертировать данные в UTF-8.

\section{ UTF-8 в PHP } \label{sect4_10}

%101

PHP начиная с версии 4 имеет встроенную поддержку Unicode. Это означает, что для работы с Unicode не потребуется никаких изменений. Но вот работа со строками потребует использование дополнительного расширения и функций из него ( всё потому, что обычные символы в php однобайтовые и функции для работы со строками считают что один байт --- один символ ).

\section{ UTF-8 в других языках } \label{sect4_11}

Большинство современных языков программирования имеют встроенную поддержку работы с Unicode  и не требуют установки каких-либо дополнительных модулей. 

%102

\section{ UTF-8 в MySQL } \label{sect4_11}

MySQL поддерживает работы в Unicode и работает так же как и с обычным ASCII текстом.

При работе в связке PHP+MySQL необходимо забыть о работе со строками на стороне базы данных - необходимо все операции проводить на стороне программы ( Если же в настройках базы данных установить, что все данные в Unicode, то тогда можно использовать функции работы со строками). Такое же предостережение применимо и к полнотекстовому индексу, но его нельзя перенести на обработку внутрь программы.

%103

MySQL, начиная с версии 4.1 полностью поддерживает работа с UTF-8, а длина строки теперь измеряется не в байтах, а в символах, что позволяет использовать все функции работы со строками.


\section{ UTF-8 в Электронной почте } \label{sect4_12}

%104,105

Использование UTF-8 обусловлена проблемой чтения письма на стороне клиента. Для того, чтобы клиент мог прочитать письмо в UTF-8 необходимо в заголовке письма указать используемую кодировку.

\section{ UTF-8 в JavaScript } \label{sect4_13}

%106

Все современные браузеры и движки JavaScript могут корректно работать с данными в UTF-8. Необходимо лишь конвертировать строки в специальный формат для их передачи методом GET.

%107

\section{ UTF-8 в API } \label{sect4_14}














 
 
 
 
 
 
 
 
 
 
 
 
 
 
 
 
 
 
 
 
 
 
 
 
 
 
 
 
 
 
 
 
 
 
 
 
 
 
 
 
 
 
 
 
 
 \clearpage

 