\chapter{ Интернализация, локализация и Unicode } \label{chapt4}

%87

Мир говорит на разных языках. Поэтому создавая приложения на одном языке сразу же теряется очень большая часть возможных пользователей.

Задача интернализации сложна и требует применения определенных знаний и методов.

\section{ Интернализация и локализация} \label{sect4_1}

 %88
 
 Интернализация приложения --- создание приложения, которое может быть использовано людьми говорящих на разных языках. Локализация приложения --- изменения приложения согласно тому где находиться пользователь.
 
 \section{ Интернализация web-приложений} \label{sect4_2}
 
 В девяностых годах прошлого тысячелетия была огромная проблема с интернализацией приложений. Каждый язык использовал свою кодировку. Поскольку для разных языков кодировка отличалась кодом букв в таблице ASCII ( между 100 и 250 ), то вывести на одной странице набор данных в разных кодировках было невозможно ( один и тот же код в разных кодировках означала разный символ).
 
 %89
 
 В 1991 году был создан Unicode --- кодировка которая содержит символы всех остальных кодировок. Таким образом один код всегда означает один символ. Существуют две основные кодировке на базе Unicode: UTF-8 и UTF-16 ( расширенная ).
 
\section{Локализация в web-приложения} \label{sect4_3}

Локализация подразумевает отображение разного контента для различных людей, согласно их местоположению.

%90

\subsection{Замена строк} \label{sect4_3_1}

Первый метод интернализации заключается в замене текста на одном языке на текст в другом языке. Для этого используется GNU приложение gettext.

\subsection{Множественные шаблоны} \label{sect4_3_2}

При правильном разделении логики и представления, можно создавать множественные шаблоны ( для каждой интернализации свой набор шаблонов). Таким образом сохраняя один и тот же функционал, можно использовать разное представления для разных людей.

%91

\subsection{Множественный интерфейс} \label{sect4_3_3}

Использование собственного интерфейса для каждого пользователя ( интернализации ) подразумевает собственную бизнес логику, логику представления и шаблоны. Таким образом  придется создавать часть приложения для каждого пользователя. Так же необходимо решить, будут ли данные пересекаться между интернализациями или каждая будет иметь свой набор уникальных данных.

\section{Замена строк} \label{sect4_4}




 
 
 
 
 
 
 
 
 
 
 
 
 
 
 
 
 
 
 
 
 
 
 
 
 
 
 
 
 
 
 
 
 
 
 
 
 
 
 
 
 
 
 
 
 
 \clearpage

 