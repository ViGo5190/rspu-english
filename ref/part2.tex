\chapter{ Архитектура web-приложений } \label{chapt2}

%24

Перед началом активной разработки необходимо продумать основные идеи архитектуры приложения. В этой главе будут рассмотрены некоторые принципы проектирования web-приложений в условиях реального мира, особенности аппаратной составляющей и её влияние на проект.

\section{ Приложение, как слоёный пирог} \label{sect2_1}

Хорошее приложение должно иметь несколько слоёв абстракций.

Низший, основной слой, --- постоянное хранилище. Данные в хранилище лежат в основе приложения. 

%26

Следующий слой это бизнес логика. Именно бизнес логика определяет как  и какие данные будут записаны и прочитаны из хранилища. Основной язык для написания бизнес логики --- PHP, или другой скриптовый язык.

Следующий уровень абстракций --- логика взаимодействий. Именно она отвечает за то, какие данные будут использованы вместе и где. Это логика не зависит от бизнес логики и тем более от логики хранилища.

Верхний уровень абстракций --- это пользовательский интерфейс, который, обычно, представлен языком разметки. Именно этот слой отображает данные пользователю в приятном для него виде.

%27
Но как истинный кулинар не оставит своё блюдо без украшения, так и нельзя оставить пользовательский интерфейс без украшений, который в web-приложениях представлены в виде CSS.

Каждый слой имеет своё предназначение и предоставляет определённый функционал. Слои должны идти в правильном порядке для полноценного взаимодействия и не перемешиваться, хоть в реальных проектах зачастую и происходит смешивание этих слоёв.

\section{ Технологии расслоения } \label{sect2_2}

Верхним слоем абстракций являются каскадные таблицы стилей. В примерах будут встречаться встроенные стили, но всё же будем стараться разделять язык разметки от стилей.

Под уровнем представления лежит уровень отображения, который основан на языке разметки. Основным языком разметки является HTML и его семейство. Непроходимо разделять логику и представление(отображение) --- для этого необходимо отдельно выносить шаблоны, которые будут заполняться логикой на основе данных из хранилища. 

%28

При разделении логики и представления, вносимые изменения в один слоёв не будут влиять на другой. Таким образом, например, изменив представления с табличного на текстовое, логика никак не измениться.

Разделение логики и представления --- один из ключевых принципов проектирование программного обеспечения.


