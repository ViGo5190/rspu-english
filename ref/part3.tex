\chapter{ Среда разработки } \label{chapt3}

%45

Разработка большого приложения с участием огромного количества разработчиков требует умения координировать действия, отслеживать кто что сделать, разрешать конфликты одновременной правки файлов, отслеживать, что приложение работает корректно. Все это проблемы будут рассмотрены в этой главе.

\section{ Три правила} \label{sect3_1}

Разработка любого уровня приложение подразумевает соблюдение трёх основных правил:
\begin{itemize}
\item Управляй исходным кодом;
\item Используй одношаговую сборку;
\item Отслеживайте свои ошибки.

\end{itemize}

\section{ Управление исходным кодом} \label{sect3_2}

Неважно, один ли разработчик в проекте или сотня --- используйте систему контроля за исходных кодом. Если же Вы до сих пор не используете систему контроля версий --- Вам стоит немедленно остановить разработку и внедрить её.

\subsection{ Что значит управлять исходным кодом?} \label{sect3_2_1}

Умение в любой момент времени вернуться к ранее сделаным правкам --- есть суть управления исходным кодом. Многие студии разработки позволяют хранить историю изменений исходных кодов, но лишь до того, как вы закроете их и потеряете всю историю изменений. Некоторые форматы, как, например, Corel WordPerfect, хранят всю историю правок внутри себя, но таким образом размер файла растёт после каждой правки, что неминуемо приведет к <<жирному>> проекту. Умение хранить историю изменений называется версионирование.

\subsection{ Версионирование } \label{sect3_2_2}

Версионирование --- одна из главных особенностей управление исходным кодом, позволяющая в любой момент времени получить любую версию исходного кода. Работа с системой контроля версий  проходит по следующему алгоритму. Сначала пользователь делает клонирование исходного кода из внешнего репозитория ( места, где хранятся все версии исходного кода ), после этого вносит правки, и, наконец, фиксирует изменения в удаленном репозитории. Таким образом в репозитории появится новая версия документа. Каждая версия документа имеет уникальный идентификатор, называемый номером ревизии. Когда пользователь клонирует к себе фалы из репозитория, он всегда получает самую новую версию файла.

%47

\subsection{ Откат изменений } \label{sect3_2_3}

В системе контроля версий не было бы смысла, не будь там возможности в любой момент времени получить любую версию файла, отменив совершенные изменения. Получать состояние системы возможно по номеру ревизии, по дате и времени существования ревизии, по тэгам и веткам. Таким образом всегда есть возможность откатить исходный код на любой состояние.

\subsection{ Журналирование } \label{sect3_2_4}

Каждый раз при фиксировании изменений, разработчик может указать комментарий ( лог ). В комментарии можно описать внесенные изменения, причины и следствия изменений. В дальнейшем это комментарии помогут лучше определить какая ревизия за что отвечает.

\subsection{ Различия } \label{sect3_2_5}

Команда \textbf{diff} позволяет отобразить различия между двумя файлами. Различия выводиться в виде строк, которые были изменены, удалены, добавлены; при это сразу видно что было и что стало. Просматривая различия совместно с логами можно лучше понять причины внесенных изменений, а так же понять кто совершил эти изменения...

%48

\subsection{ Одновременная правка и слияния } \label{sect3_2_6}

Одновременное редактирование проекта подразумевает, что каждый редактор будет иметь своего пользователя в системе контроля версий. Таким образом можно журналировать все внесенные изменения конкретным пользователем.

Если два пользователя одновременно редактировали один и тот же файл, и попытались зафиксировать внесенные изменения, то неминуемо придётся выполнять слияние двух правок. Если же пользователи редактировали разные участки одного файла, то большинство система контроля версий смогут в автоматическом режиме разрешить все возникшие конфликты. Если же правился один и тот же участок кода, то неминуема ручное слияние.

На практике конфликты встречаются очень редко. Обычно они возникают если разработчики не часто обновляют свои рабочее версии с репозитория. Или же два разработчика работают над одним и тем же кодом, что говорит о плохой коммуникацией между ними.

Для версионирования бинарных файлов или изображений может потребоваться использование специализированных средства контроля версий.

\subsection{ Аннотации } \label{sect3_2_7}

Большинство систем контроля версий позволяют пользователю просмотреть данные по каждой строчке, где указаны время изменения, автор изменения, номер ревизии.

%49

Просмотр аннотации позволяет определить ревизию, где были внесены правки. По этим данным можно просмотреть лог фиксирования изменения, не просматривая каждую ревизию.

\subsection{ Обсуждение блокировок } \label{sect3_2_8}

Система контроля версий подразумевает, что любой пользователь может в любой момент времени редактировать любой файл. Но иногда возникает необходимость запретить другим пользователям вносить правки в определенный файл на время редактирования этого файла кем-то.

Блокировки несут два проблемы, с одной стороны одни полностью помогают избежать конфликтов, но не дают одновременной правки разных частей одного и того же документа, где не может быть конфликтов. Вторая проблема связана с тем, что кто-то может забыть снять блокировку с файла и тогда файл может остаться заблокированным надолго.

%50

\subsection{ Проекты и модули } \label{sect3_2_9}

Системы контроля версий позволяют группировать файлы и директории, превращая несколько файлов в один проект или модуль. Таким образом можно отслеживать изменения не только одиночного файла, но и всего проекта.

 \subsection{ Тэгирование } \label{sect3_2_10}
 
 Использование тегов позволяет делать снимок всего репозитория, то есть находясь в определенной ревизии, создание тега, позволит сохранить состояние всех файлов в репозитории на данный момент. Таким образом очень удобно помечать релизное состояние репозитория.
 
  \subsection{ Ветвление } \label{sect3_2_11}
  
  Последняя ревизии обычно называется головой. Иногда возникает необходимости в наличии больше двух голов. Для необходимо создать новую ветку в репозитории. То есть репозиторий создаст свою точную копию, называемую веткой. Можно работать и фиксировать изменения в каждую из ветвей.
  
Основная ветвь называется стволом. Именно в ней хранятся стабильные релизы. Остальные же ветви используются на создание нового функционала. Особенно это оправдывает себя, когда несколько разработчиков работают одновременно над разными новшествами в разным частях проекта. Ветвление помогает им не заботиться о поддержке кода, написанного другими разработчиками, а так же не зависеть от того, что другие разработчики зафиксировали в репозитории.

  \subsection{ Слияние } \label{sect3_2_12}
  
  Когда разработка нового функционала в отдельной ветке закончено, необходимо перенести все эти изменения в основную ветку. Это действие принято называть слиянием.
  
 %51
 
 Процесс слияния, так же как и фиксирования. требует  разрешения конфликтов. Чаще всего это происходит автоматически. Но иногда приходиться производить слияние и в ручном режиме.

Принято считать, что основная ветка --- очень особенная ветка. она никогда не вливается в другую. Лишь все ветки вливаются в неё.


\section{ Необходимые инструменты} \label{sect3_3}

Есть некоторые необходимые инструменты, которые позволяют улучшить результаты.

\subsection{ Командная строка и интеграция редактирования } \label{sect3_3_1}

Большинство систем контроля версий, так же как и множество продуктов распространяемых с открытым кодом, используют лишь режим работы в командной строке, что требует специфических навыков и знаний.

Современные операционный системы предлагают возможность работы с системой контроля версий по средствам графического интерфейса. Так же многие современные средства разработки имеют встроенную поддержку систем контроля версий ( или функционал расширяется с помощью дополнений).

Использование хороших редакторов  с интеграцией в командной строке позволит свободно пользоваться функционалом систем контроля версий даже тем разработчикам, кто плохо знаком с командной строкой.

%52

\subsection{ Web интерфейс } \label{sect3_3_2}

Большинство систем управления предоставляют удобный web-интерфейс, где в интерактивном режиме можно просматривать различные ветки, ревизии, каталоги и файлы. Web-интерфейс дает возможность использования всех возможностей систем контроля версий тем, кто не владеет навыками работы в консольном режиме.

\subsection{ Отправка отчетов о фиксировании изменений на почту } \label{sect3_3_2}

Отправка отчетов о произведенных фиксированиях позволяет отслеживать и моментально реагировать на все произошедшие изменения. Так менеджер проекта может сразу понять кто и куда внёс правки. А благодаря тому, что в письмо можно включить и произведенные изменения ---- можно отслеживать все события не заходя в систему контроля версий.

\subsection{ Отображение отчетов о фиксировании изменений в RSS ленту } \label{sect3_3_3}

 Использование RSS ленты аналогично использованию рассылки отчетов фиксирования.  Различия заключаются в том, что длина ленты ( кол-во отчетов ) ограничено, что не даст посмотреть старый изменения.
 
 %53
 
 \subsection{ Использование СУДБ для хранения информации о фиксированиях } \label{sect3_3_4}

Использование СУБД для хранения информации о фиксированиях позволяет строить более подробные отчеты по данным фиксироваий. Так можно с помощью таких данных и возможностей СУДБ быстро установить все ревизии, в которых были проведены изменения в конкретном файле.

 \subsection{ Действия после фиксирования } \label{sect3_3_5}
 
 Системы контроля версий дают возможность выполнять различные действия после фиксирования. Так, например, можно запустить тестирование основной ветки сразу после фиксирования в неё новых изменений.
 
 %54
 
\section{ Системы управления исходным кодом } \label{sect3_4}
 
Существует множество систем управления исходным кодом. Разные системы предоставляют различные дополнения и возможности. При выборе системы контроля версий необходимо так же учитывать наличие клиентов для операционной системы в которой производиться разработка, необходимость web-интерфейса, списка рассылок и прочее.

 \subsection{ The Revision Control Sustem (RCS) } \label{sect3_4_1}
 
 Предшественником современных систем контроля версий является RCS, разработанная Волтером Тичи в университете Пердью в начале восьмидесятых 20 века.
 
 RCS нельзя назвать полноценной системой контроля версий, ибо она предназначена для хранения ревизий одного файла, не имеет возможности удаленного использования, не поддерживает работу нескольких пользователей.
 
\subsection{ The Concurrent Versions System (CVS) } \label{sect3_4_2}

CVS была представлена в 1986 году Диком Груне в Брюссельском Свободном Университете, как система контроля версий для хранения больших проектов с большим количеством разработчиков и возможностью использования удаленных репозиториев.  CVS позволяет использовать удаленное управление, поддерживает работу  тегами, ветвление, слияния. блокировки и обычный функционал.

\textbf{Наличие клиента.} Поскольку CVS существует довольно давно, существуют клиенты под все операционные системы. Так же есть не только консольные клиенты, но и клиенты с графическим интерфейсом.

Так же большинство современных сред разработки имеют встроенную возможность работы с CVS.

\textbf{Web-интерфейс.} CVS имеет два основных web-интерфейса, которые очень похожи внешне и предоставляют базовый функционал. Обе системы возможно расширять за счет плагинов.

\textbf{Список расссылки и RSS лента.} CVS поддерживает плагины, поэтому существуют расширения,  которые позволяют реализовать функционал рассылки писем послее фиксирования изменений, а так же публиковать изменения в RSS ленту.

Система расширений позволяет реализовать запуск тестирования после каждого фиксирования и прочие возможности.

\textbf{База данных фиксирований.} CVS поддерживает Bonsai --- база данных для фиксирований, созданная Терри Вейнссманом в качестве проекта Mozilla и написана на Perl. Система реализует весь функционал баз данных фиксирований.

\textbf{Плюсы.}
\begin{itemize}
\item Бесплатно --- говорит само за себя;
\item Используется многими и проверено временем;
\item Доступны реализации клиентов и серверов для всех платформ.
\end{itemize}

\textbf{Минусы.}
\begin{itemize}
\item Версионирование на уровне файлов, а не репозитория --- CVS не предоставляет возможности зафиксировать изменения нескольких файлов одним фиксированием. Каждый файл получит разный номер ревизии, тем самым нарушается атомарность операции фиксирования;
\item Невозможно переименовать или переместить файл --- поскольку CVS хранит история ревизий каждого файла в отдельном файле, невозможно переименовать или переместить файл. Для переименования файла необходимо удалить старый и создать новый, со старым содержимым, что неминуемо приведет к потере всей истории.
\end{itemize}

%57

\subsection{ Subversion(SVN) } \label{sect3_4_3}

Subversion --- это проект с открытым исходным кодом, который стартовал в 2000 году с четкой целью исправит все ошибки VCS. 

Subversion, в отличии от VCS, хранит всю историю фиксирований в базе данных, что позволяет избежать основных проблем VCS.

\textbf{Наличие клиента.} Subversion появляется на всё больших операционных системах. Есть графические интерфейсы, интеграции в различные среды разработки.

\textbf{Web-интерфейс.} Есть несколько хороших решений для  работы с этой системой контроля версий через интернет. Они представлены websvn, trac и Chora (которая так же поддерживает CVS).

\textbf{Список расссылки и RSS лента.} Subversion так же как и CVS предоставляет механизм, который позволяет после успешного фиксирования запускать различные расширение, которые в свою очередь могут отправлять изменения на почту или в RSS ленту.

\textbf{База данных фиксирований.} Subversion имеет клон Bonsai, названый Kamikaze, который написан на Perl и использует MySQL в качестве основного хранилища.

\textbf{Плюсы.}
\begin{itemize}
\item Бесплатно --- говорит само за себя;
\item Атомарность операции фиксирования;
\item Позволяет перемещать и переименовывать файлы.
\end{itemize}

%58

\textbf{Минусы.}
\begin{itemize}
\item Ненадёжное хранилище --- по умолчанию Subversion использует в качестве хранилища BerkleyDB. При падении Subversion может сломать и BerkleyDB, что приведет к повреждению хранилища. Начиная с версии 1.2 Subversion может использовать в качестве хранилища FSFS, которая решает проблемы надежности BerkleyDB.

\item Тяжело скомпилировать --- Subversion тяжело собрать из исходных кодов, поэтому сейчас она поставляется в виде скомпилированных бинарных файлов.
\end{itemize}


\subsection{ Perforce } \label{sect3_4_4}

Perforce --- коммерческий продукт, реализующий систему контроля версий. Естественно он не бесплатный, но за плату так же можно получить техническую поддержку, что является плюсом для многих компаний.

Perforce использует ту же идеологию в названиях, что и остальные системы контроля версий, только репозитории называются складами.

%59

\textbf{Наличие клиента.} Perforce имеет собственный клиент, который обычно и выбирают разработчики. Графический клиент доступен для ОС Windows, который позволяет кроме обычных функций возможность просматривать репозитории и отслеживать изменения. Perforce имеет API, который позволяет писать плагины для различных систем.

\textbf{Web-интерфейс.} Perforce имеет собственный web-интерфейс для просмотра репозиториев, который предоставляет функционал схожий с CVS и Subversion. Он бесплатен для всех пользователей Perforce.

\textbf{Список расссылки и RSS лента.} Perforce так же как и CVS и Subversion предоставляет механизм, который позволяет после успешного фиксирования запускать различные расширение, которые в свою очередь могут отправлять изменения на почту или в RSS ленту.

\textbf{База данных фиксирований.} Поддержка Perforce в Bonsai лишь запланирована.

\textbf{Плюсы.}
\begin{itemize}
\item Атомарность операции фиксирования;
\item Система прав доступа;
\item Коммерческая поддержка.
\end{itemize}



\textbf{Минусы.}
\begin{itemize}
\item Стоимость лицензии для одного пользователя составляет от 500 до 800 долларов. Проекты с открытым исходным кодом могут пользоваться системой бесплатно.
\end{itemize}

%60


\subsection{ Visual Source Safe (VSS) } \label{sect3_4_5}

VSS --- система контроля версий компании Microsoft. Ранее это был отдельный продукт, но ныне входит в в состав Visual Studio. В прошлом VSS могла работать только в проприаретарном режиме, блокирую доступ к файлу всем, кроме того, кто его редактирует; но сейчас поддерживается одновременное редактирование несколькими пользователями.

\textbf{Наличие клиента.} Официальный клиент доступен только на ОС Windows. Так же лишь немного среды разработки имеют поддержку VSS. Есть неофициальные реализации клиент для других операционных систем.

\textbf{Web-интерфейс.} Заявлено о наличии web-интерфейса, но он до сих пор не представлен и информации о нём крайне мало.

\textbf{Список расссылки и RSS лента.} На данный момент не реализованы возможности отправки изменений на почту или в ленту RSS.

\textbf{База данных фиксирований.} На данный момент отсутствуют какие-либо реализации.

\textbf{Плюсы.}
\begin{itemize}
\item Лёгкая интеграция в Visual Studio.
\end{itemize}



\textbf{Минусы.}
\begin{itemize}
\item Отсутствие атомарных списков изменений;
\item Отсутствие возможности переместить или переименовать файл;
\item Отсутствие аннотаций или пометок о том, кто сделал правку;
\item Отсутствие полноценной поддержки работы по сети.
\end{itemize}

\subsection{ В заключении } \label{sect3_4_6}

Представленные четыре продукта являются широко используемым на данный момент в разработке web-приложений.

\subsection{ Итоги } \label{sect3_4_7}

На данный момент самый лучший выбор --- Subversion, ибо она реализует наибольшее количество возможностей, а так же поддерживает сторонние приложения.

Если вы уже используете систему контроля версий и она исправно работает, то не стоит её менять, ибо переезд может вызвать множество проблем как с конвертированием репозиториев, так и с применением клиентов в различных ОС.

Конечно, если нужна поддержка, можно использовать и Perforce, но поработав с Subversion, можно понять, что поддержка не нужна.

%62

\section{ Что хранить в системе контроля версий } \label{sect3_5}

Обычно в системе контроля версий хранится исходный код, статичный файлы и изображения. Но это не единственный подход для организации хранения данных в системах контроля версий.

\subsection{ Документация } \label{sect3_5_1}

Хранения документации проекта в системе контроля версий позволяет решить несколько проблем и улучшить процесс разработки.

Так при одновременной правке исходников и документации с последующим фиксированием позволяет иметь актуальную документацию в любой момент, а это означает, что любая ревизия будет содержать верную документацию, относящуюся именно к текущему состоянию проекта.

Документация рядом с исходным кодом позволяет не тратить время на поиски документации. Так же документация в системе контроля версий, при наличии web-интерфейса, позволяет получить доступ к ней в любом месте даже там, где нету возможности выкачать исходный код из системы контроля версий.

\subsection{ Конфигурационные файлы программного обеспечения } \label{sect3_5_2}

%63

В Unix подобных системах конфигурация приложения хранится в обычны файлах. Поэтому их очень удобно хранить в системах контроля версий, рядом с исходным кодом. Таким образом мы получим актуальные настройки  в любой ревизии.

Конфигурационные файлы в системе контроля версий помогут быстрее развернуть проект на новом оборудовании и вернуть в строй на текущем сервере после некоторой поломки.

\subsection{ Инструменты постройки проекта } \label{sect3_5_3}

Все скрипты, которые помогают собирать проект, должны так же хранится в репозитории. Надо взять правило, что всё что может быть изменено хотя бы раз --- должно находиться в репозитории.

\section{ Что не стоит хранить в системе контроля версий } \label{sect3_6}

Если проект компилирует исходники в бинарные файлы, то в репозиторий имеет смысл складывать только исходный код, скрипты для сборки и инструкции. Бинарные файлы в репозитории не могут быть сравнены между разными версиями, что в корне нарушает всю идеологию систем контроля версий.

%64

\section{ Сборка проекта в один шаг } \label{sect3_7}

По мере роста проекта и внедрения нового функционала процесс сборки становится многошаговым и требует выполнения различных действия в разных местах. Такой процесс может сломаться на каком-то из шагов, что может привести к катастрофе. Поэтому необходимо всё автоматизировать и привести процесс сборки проекты к нажатию одной кнопки, которая сделать все необходимые действия.

\section{ Резать по живому } \label{sect3_8}

В начале разработки проект живет на локальной машине и все вносимые правки сразу видны. Но когда проект переезжает на удаленную машину, для внесения правок в живом проекте необходимо подключаться к серверу и вносить правки в код, который там расположен. С одной стороны, все внесенные правки сразу отобразятся, с другой стороны, если допустить ошибки --- все пользователи сразу увидят эти ошибки.

\section{ Создание рабочего окружения } \label{sect3_9}

Для полноценной разработки проекта необходимо наличие трёх сред окружения:
\begin{itemize}

\item \textbf{Dev} --- разработка;

\item \textbf{Staging} --- тестовое окружение;

\item \textbf{Production} --- рабочее окружение.

\end{itemize}

%65

\subsection{ Окружение для разработки } \label{sect3_9_1}

В окружении для разработки разработчики могут вести любую деятельность. Создание и тестирование новых улучшений, исправления текущих проблем и прочее. Это окружение часто копирует рабочее окружение, но с меньшими наборами данных.

Окружения для разработки может находиться на выделенном сервере, тогда все разработчики работают там вместе, что может вызывать проблемы. Часто каждый разработчик настраивает рабочее окружение на своём рабочем месте и работает в нём, что даёт полную автономность и независимость от правок других разработчиков.

%66

\subsection{ Тестовое окружение } \label{sect3_9_2}

Тестовое окружение подразумевает то же оборудование и набор данных, что и рабочее окружение. Что же касается исходного кода, то в тестовое окружение выноситься определённый снимок проекта ( то есть если разработчики внесли правки в репозиторий, то они не появятся автоматически на тестовом окружении ).

Иногда возникает необходимость  разрабатывать некоторые улучшения в использованием данных с рабочего окружения. Для этого создаётся ещё одно окружение, которые может работать с данными из рабочего окружения, но во всём остальном является окружением для разработки.

\subsection{ Рабочее окружение } \label{sect3_9_3}

Рабочее окружение --- единственное окружение которые видят пользователи. После полного тестирования на тестовом окружении код попадает сюда.

Иногда требуется протестировать новый функционал на группе пользователей. Для этого создаётся дополнительное рабочее окружение. Для полноценной рабочего дополнительного рабочего окружения требуется так же наличие дополнительного тестового окружения.

%67

\section{ Процесс создания релизов } \label{sect3_10}

Процесс создания релизов можно разделить на шаги.

\begin{itemize}

\item \textbf{Разработка.} На этом этапе разрабатывается новый функционал, исходный код загружается на окружения для разработки. Ищутся ошибки и исправляются. В конце этапа код подготавливается для полного тестирования.

\item \textbf{Фиксирование и тестирование.}  Изменения кода фиксируются в системе контроля версий. Потом попадает на тестовое окружение, где и тестируется.

\item \textbf{ Развертывание.} Код переносится со тестового окружения в рабочее. Пользователи начинают использовать новый особенности. 

\end{itemize}

Последние два этапа часто поддаются полной автоматизации.

\section{ Инструменты для сборки } \label{sect3_11}

Инструменты для сборки обычно представляют из себя несколько скриптов и программ, которые выполняются на втором и третьем шаге в процессе релиза.

%68

При вытаскивании кода из системы контроля версий иногда требуется поменять пути до программ, а так же установить права на папки. По сути это и есть конфигурационные файлы. Таким образом для упрощения разработки можно создать  набора конфигураций : для среды разработки и рабочей среды.  И тогда и при разворачивании проекта будет достаточно один конфигурационный файл заменить другим.

%69

В процессе развертывании проекта может потребоваться перезагрузка сервисов в связи со сменой конфигурационных файлов.

Хорошей практикой является хранения специфичных конфигурационных файлов сервера на самом сервере.

 %70
 
 Такая практика поможет отслеживать ошибки, ведь в журнале событий будет указан номер сервера.
 
 Так же необходимо что бы пользователь который начал работу со старой версий продукта --- закончил работу с ней же, в противном случае он получит неожиданный результат.
 
 \section{ Управления релизами } \label{sect3_12}
 
 В команде разработчиком необходим специально обученный человек, который отвечает за тестирование, разворачивание и отслеживание ошибок в процессе релиза. Тогда разработчик по окончанию работы над новым функционалом должен обратиться к менеджеру релизов, который запустит процесс релиза.
 
 В небольшой команде каждый разработчик может отследить и проверить изменения других ,после чего запустить процесс релиза.
 
 Разработка каждого нового улучшения в одной ветке позволяет избежать проблем, появляющихся при работе всех в одной ветке, ведь разработчик может четко отслеживать свою ветку и тестировать. После завершения разработки ветка вливается в ствол.
 
 %71
 
 Можно держать новый функционал и в стволе проекта, но для этого надо настроить конфигурацию таким образом, что бы в режиме разработке функционал работал, а во всех остальных он был выключен.
 
 Оба подхода ( работа в ветках и работа в стволе ) имеют права на жизнь. Но большая команда разработчиков должна придерживаться одного их подходов, хотя проблем в развертыванием не избежать. Для полноценного развертывания можно завести специализированную ветку в которой каждая ревизия будет являться кандидатом для рабочего окружения.

 %72
   
\section{ Что не стоит автоматизировать} \label{sect3_13}

\subsection{ Изменения структуры Базы Данных } \label{sect3_13_1}

При изменении базы данных надо понимать, что, во-первых, это может занять продолжительное время, во-вторых, это не всегда возможно отменить.

Изменения в базе данных могут затронуть множество серверов и занять продолжительное время. Кроме того необходимо понимать, что изменения могут заблокировать базу данных и приложение перестанет работать.

Проблема отката совершенных изменений упирается в объем хранимых данным. Если при изменения удаляются столбцы или строки таблицы, то эти данные будут потерянны навсегда, поэтому необходим механизм резервного копирования.

Чтобы избежать проблем при изменении структуры базы данных необходимо отказаться от автоматического обновления. Необходим скрипт, который будет сравнивать структуры базы в рабочем окружении и окружении разработки, после чего выделять изменения. Эти изменения будет полезно сохранить на случай возможно отката.  После применения изменений необходимо запустить ещё один скрипт который проверит целостность базы и сравнит рабочую версию с версией разработчиков.

%73

\subsection{ Изменения конфигурации программного и аппаратного обеспечения } \label{sect3_13_2}

Обновления конфигурационных файлов для программного и аппаратного обеспечения не должно выполнятся в автоматическом режиме. Эти действия должны быть апробированы на тестовых серверах и лишь потом выполняться на рабочих.

Необходимо разделить процесс развёртывания приложения и процесс обновления конфигураций, что позволит разработчикам заниматься лишь своей задачей.

В идеальном мире будут два тестовых окружения и два окружения для разработки: в одном будет разрабатываться приложения, во втором конфигурация программного обеспечения.

\section{ Отслеживание проблем } \label{sect3_14}

Отслеживание ошибок и проблем является обязательной частью разработки приложения.

%74

Идеальным решениям для отслеживания проблем и ошибок будет web-приложение, к которому все ( не только разработчики ) будут иметь доступ. 

\section{ Минимальный набор функций } \label{sect3_15}

Хорошее ПО для отслеживания ошибок позволяет в первую очередь отслеживать найденные ошибки. Но есть и дополнительный функции. Можно комментировать ошибки. Когда добавлена новая ошибка, видно кто её создали  и можно понять кто из разработчиков взялся за её решение. Есть возможность отслеживать решение проблемы. Так же многие программа позволяют отсылать письма при изменении статуса проблемы.

\section{ ПО для отслеживания ошибок } \label{sect3_16}

При начале разработке необходимо выбрать ПО с помощью которого можно будет отслеживать проблемы и ошибки.

\subsection{ FogBugz } \label{sect3_16_1}

FogBugz созданный компанией Fog Creek Software является платным, но не дорогим ( сто долларов за разработчика ). FogBugz имеет две версии, одна написана для Windows , вторая для Unix  подобных ОС.

\texbf{ Плюсы}

\begin{itemize}

\item Очень простая в использовании;

\item Активно разрабатывается и имеет поддержку;

\item Данные хранятся в MySQL.

\end{itemize}

\texbf{ Минусы}

\begin{itemize}

\item Платная;

\item Отсутствие дополнительного функционала.

\end{itemize}

\subsection{ Mantis Bug Tracker } \label{sect3_16_2}

Mantis Bug Tracker --- по с открытым исходным кодом, написанное на PHP, хранящее данные в MySQL. На данный момент ожидается релиз версии 1.0

\texbf{ Плюсы}

\begin{itemize}

\item Легко установить;

\item Написано на PHP  и использует  MySQL;

\item Активное сообщество разработчиков;

\item Поддержка интеграции с CVS.

\end{itemize}

%76

\texbf{ Минусы}

\begin{itemize}

\item Скромный функционал;

\item Недокументированный конфигурационный файл.

\end{itemize}

\subsection{ Request Tracker ( RT ) } \label{sect3_16_3}

RT создан в 1996 году. Написан на Perl. Поддерживает все основные базы данных для хранения своих данных.

\texbf{ Плюсы}

\begin{itemize}

\item Поддержка создания проблем через электронную почту;

\item Поддерживает большое количество баз данных;

\item Использует простые шаблоны, которые можно изменить под свои требования.

\end{itemize}

%76

\texbf{ Минусы}

\begin{itemize}

\item Ориентированность на электронную почту упрощает создание проблем, но усложняет процесс отслеживания;

\item Чрезмерно сложный процесс просмотра и управления;

\item Интерфейс созданный разработчиками позволяет использовать множество возможностей, но тяжело настраивается.

\end{itemize}


\subsection{ Bugzilla } \label{sect3_16_4}

Bugzilla было создана Mozilla Foundation, которая разрабатывает одноименный браузер. Проект вышел в 1998 году, на данный момент имеет широкий функционал и активное сообщество разработчиков.

\texbf{ Плюсы}

\begin{itemize}

\item Очень большое число возможностей;

\item Использует MySQL;

\item Проверено огромным числом людей.

\end{itemize}

%77

\texbf{ Минусы}

\begin{itemize}

\item Слишком большое количество возможностей;

\item Невозможно запустить как часть Apache, только в CGI режиме;

\item Невозможно добавить новые поля к проблеме.

\end{itemize}

\subsection{ Trac } \label{sect3_16_5}

Trac написан на Python и имеет встроенную интеграцию в SVN. Основное хранилище - SQLite.

\texbf{ Плюсы}

\begin{itemize}

\item Легко устанавливается;

\item Честный интерфейс.

\end{itemize}

\texbf{ Минусы}

\begin{itemize}

\item Использование SQLite не позволяет интегрироваться с приложением;

\item Работает только с Subverion;

\item Требует шаблонизатор ClearSilver.

\end{itemize}


\section{ Что отслеживать? } \label{sect3_17}

\subsection{ Ошибка } \label{sect3_17_1}






 
 
 

 



































\clearpage



