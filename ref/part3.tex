\chapter{ Среда разработки } \label{chapt3}

%45

Разработка большого приложения с участием огромного количества разработчиков требует умения координировать действия, отслеживать кто что сделать, разрешать конфликты одновременной правки файлов, отслеживать, что приложение работает корректно. Все это проблемы будут рассмотрены в этой главе.

\section{ Три правила} \label{sect3_1}

Разработка любого уровня приложение подразумевает соблюдение трёх основных правил:
\begin{itemize}
\item Управляй исходным кодом;
\item Используй одношаговую сборку;
\item Отслеживайте свои ошибки.

\end{itemize}

\section{ Управление исходным кодом} \label{sect3_2}

Неважно, один ли разработчик в проекте или сотня --- используйте систему контроля за исходных кодом. Если же Вы до сих пор не используете систему контроля версий --- Вам стоит немедленно остановить разработку и внедрить её.

\subsection{ Что значит управлять исходным кодом?} \label{sect3_2_1}

Умение в любой момент времени вернуться к ранее сделаным правкам --- есть суть управления исходным кодом. Многие студии разработки позволяют хранить историю изменений исходных кодов, но лишь до того, как вы закроете их и потеряете всю историю изменений. Некоторые форматы, как, например, Corel WordPerfect, хранят всю историю правок внутри себя, но таким образом размер файла растёт после каждой правки, что неминуемо приведет к <<жирному>> проекту. Умение хранить историю изменений называется версионирование.

\subsection{ Версионирование } \label{sect3_2_2}

Версионирование --- одна из главных особенностей управление исходным кодом, позволяющая в любой момент времени получить любую версию исходного кода. Работа с системой контроля версий  проходит по следующему алгоритму. Сначала пользователь делает клонирование исходного кода из внешнего репозитория ( места, где хранятся все версии исходного кода ), после этого вносит правки, и, наконец, фиксирует изменения в удаленном репозитории. Таким образом в репозитории появится новая версия документа. Каждая версия документа имеет уникальный идентификатор, называемый номером ревизии. Когда пользователь клонирует к себе фалы из репозитория, он всегда получает самую новую версию файла.

%47

\subsection{ Откат изменений } \label{sect3_2_3}

В системе контроля версий не было бы смысла, не будь там возможности в любой момент времени получить любую версию файла, отменив совершенные изменения. Получать состояние системы возможно по номеру ревизии, по дате и времени существования ревизии, по тэгам и веткам. Таким образом всегда есть возможность откатить исходный код на любой состояние.

\subsection{ Журналирование } \label{sect3_2_4}

Каждый раз при фиксировании изменений, разработчик может указать комментарий ( лог ). В комментарии можно описать внесенные изменения, причины и следствия изменений. В дальнейшем это комментарии помогут лучше определить какая ревизия за что отвечает.

\subsection{ Различия } \label{sect3_2_5}

Команда \textbf{diff} позволяет отобразить различия между двумя файлами. Различия выводиться в виде строк, которые были изменены, удалены, добавлены; при это сразу видно что было и что стало. Просматривая различия совместно с логами можно лучше понять причины внесенных изменений, а так же понять кто совершил эти изменения...

%48

\subsection{ Одновременная правка и слияния } \label{sect3_2_6}

Одновременное редактирование проекта подразумевает, что каждый редактор будет иметь своего пользователя в системе контроля версий. Таким образом можно журналировать все внесенные изменения конкретным пользователем.

Если два пользователя одновременно редактировали один и тот же файл, и попытались зафиксировать внесенные изменения, то неминуемо придётся выполнять слияние двух правок. Если же пользователи редактировали разные участки одного файла, то большинство система контроля версий смогут в автоматическом режиме разрешить все возникшие конфликты. Если же правился один и тот же участок кода, то неминуема ручное слияние.

На практике конфликты встречаются очень редко. Обычно они возникают если разработчики не часто обновляют свои рабочее версии с репозитория. Или же два разработчика работают над одним и тем же кодом, что говорит о плохой коммуникацией между ними.

Для версионирования бинарных файлов или изображений может потребоваться использование специализированных средства контроля версий.

\subsection{ Аннотации } \label{sect3_2_7}

Большинство систем контроля версий позволяют пользователю просмотреть данные по каждой строчке, где указаны время изменения, автор изменения, номер ревизии.

%49

Просмотр аннотации позволяет определить ревизию, где были внесены правки. По этим данным можно просмотреть лог фиксирования изменения, не просматривая каждую ревизию.

\subsection{ Обсуждение блокировок } \label{sect3_2_8}

Система контроля версий подразумевает, что любой пользователь может в любой момент времени редактировать любой файл. Но иногда возникает необходимость запретить другим пользователям вносить правки в определенный файл на время редактирования этого файла кем-то.

Блокировки несут два проблемы, с одной стороны одни полностью помогают избежать конфликтов, но не дают одновременной правки разных частей одного и того же документа, где не может быть конфликтов. Вторая проблема связана с тем, что кто-то может забыть снять блокировку с файла и тогда файл может остаться заблокированным надолго.

%50

\subsection{ Проекты и модули } \label{sect3_2_9}

Системы контроля версий позволяют группировать файлы и директории, превращая несколько файлов в один проект или модуль. Таким образом можно отслеживать изменения не только одиночного файла, но и всего проекта.

 \subsection{ Тэгирование } \label{sect3_2_10}
 
 Использование тегов позволяет делать снимок всего репозитория, то есть находясь в определенной ревизии, создание тега, позволит сохранить состояние всех файлов в репозитории на данный момент. Таким образом очень удобно помечать релизное состояние репозитория.
 
  \subsection{ Ветвление } \label{sect3_2_11}
  
  Последняя ревизии обычно называется головой. Иногда возникает необходимости в наличии больше двух голов. Для необходимо создать новую ветку в репозитории. То есть репозиторий создаст свою точную копию, называемую веткой. Можно работать и фиксировать изменения в каждую из ветвей.
  
Основная ветвь называется стволом. Именно в ней хранятся стабильные релизы. Остальные же ветви используются на создание нового функционала. Особенно это оправдывает себя, когда несколько разработчиков работают одновременно над разными новшествами в разным частях проекта. Ветвление помогает им не заботиться о поддержке кода, написанного другими разработчиками, а так же не зависеть от того, что другие разработчики зафиксировали в репозитории.

  \subsection{ Слияние } \label{sect3_2_12}
  
  Когда разработка нового функционала в отдельной ветке закончено, необходимо перенести все эти изменения в основную ветку. Это действие принято называть слиянием.
  
 %51
 
 Процесс слияния, так же как и фиксирования. требует  разрешения конфликтов. Чаще всего это происходит автоматически. Но иногда приходиться производить слияние и в ручном режиме.

Принято считать, что основная ветка --- очень особенная ветка. она никогда не вливается в другую. Лишь все ветки вливаются в неё.


\section{ Необходимые инструменты} \label{sect3_3}

Есть некоторые необходимые инструменты, которые позволяют улучшить результаты.

\subsection{ Командная строка и интеграция редактирования } \label{sect3_3_1}

Большинство систем контроля версий, так же как и множество продуктов распространяемых с открытым кодом, используют лишь режим работы в командной строке, что требует специфических навыков и знаний.

Современные операционный системы предлагают возможность работы с системой контроля версий по средствам графического интерфейса. Так же многие современные средства разработки имеют встроенную поддержку систем контроля версий ( или функционал расширяется с помощью дополнений).

Использование хороших редакторов  с интеграцией в командной строке позволит свободно пользоваться функционалом систем контроля версий даже тем разработчикам, кто плохо знаком с командной строкой.

%52

\subsection{ Web интерфейс } \label{sect3_3_2}

Большинство систем управления предоставляют удобный web-интерфейс, где в интерактивном режиме можно просматривать различные ветки, ревизии, каталоги и файлы. Web-интерфейс дает возможность использования всех возможностей систем контроля версий тем, кто не владеет навыками работы в консольном режиме.

\subsection{ Отправка отчетов о фиксировании изменений на почту } \label{sect3_3_2}

Отправка отчетов о произведенных фиксированиях позволяет отслеживать и моментально реагировать на все произошедшие изменения. Так менеджер проекта может сразу понять кто и куда внёс правки. А благодаря тому, что в письмо можно включить и произведенные изменения ---- можно отслеживать все события не заходя в систему контроля версий.

\subsection{ Отображение отчетов о фиксировании изменений в RSS ленту } \label{sect3_3_3}

 Использование RSS ленты аналогично использованию рассылки отчетов фиксирования.  Различия заключаются в том, что длина ленты ( кол-во отчетов ) ограничено, что не даст посмотреть старый изменения.
 
 \subsection{ Использование СУДБ для хранения информации о фиксированиях } \label{sect3_3_4}




































\clearpage


