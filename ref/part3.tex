\chapter{ Среда разработки } \label{chapt3}

%45

Разработка большого приложения с участием огромного количества разработчиков требует умения координировать действия, отслеживать кто что сделать, разрешать конфликты одновременной правки файлов, отслеживать, что приложение работает корректно. Все это проблемы будут рассмотрены в этой главе.

\section{ Три правила} \label{sect3_1}

Разработка любого уровня приложение подразумевает соблюдение трёх основных правил:
\begin{itemize}
\item Управляй исходным кодом;
\item Используй одношаговую сборку;
\item Отслеживайте свои ошибки.

\end{itemize}

\section{ Управление исходным кодом} \label{sect3_2}

Неважно, один ли разработчик в проекте или сотня --- используйте систему контроля за исходных кодом. Если же Вы до сих пор не используете систему контроля версий --- Вам стоит немедленно остановить разработку и внедрить её.

\section{ Что значит управлять исходным кодом?} \label{sect3_3}

Умение в любой момент времени вернуться к ранее сделаным правкам --- есть суть управления исходным кодом. Многие студии разработки позволяют хранить историю изменений исходных кодов, но лишь до того, как вы закроете их и потеряете всю историю изменений. Некоторые форматы, как, например, Corel WordPerfect, хранят всю историю правок внутри себя, но таким образом размер файла растёт после каждой правки, что неминуемо приведет к <<жирному>> проекту. Умение хранить историю изменений называется версионирование.

\section{ Версионирование } \label{sect3_3}


