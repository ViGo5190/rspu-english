\chapter{ Среда разработки } \label{chapt3}

%45

Разработка большого приложения с участием огромного количества разработчиков требует умения координировать действия, отслеживать кто что сделать, разрешать конфликты одновременной правки файлов, отслеживать, что приложение работает корректно. Все это проблемы будут рассмотрены в этой главе.

\section{ Три правила} \label{sect3_1}

Разработка любого уровня приложение подразумевает соблюдение трёх основных правил:
\begin{itemize}
\item Управляй исходным кодом;
\item Используй одношаговую сборку;
\item Отслеживайте свои ошибки.

\end{itemize}

\section{ Управление исходным кодом} \label{sect3_2}

Неважно, один ли разработчик в проекте или сотня --- используйте систему контроля за исходных кодом. Если же Вы до сих пор не используете систему контроля версий --- Вам стоит немедленно остановить разработку и внедрить её.

\section{ Что значит управлять исходным кодом?} \label{sect3_3}

